\chapter{Wstęp}

\section{Ogólnie}

Praca przedstawia wykorzystanie bibliotek OpenCV, MediaPipe oraz metod 
uczenia maszynowego bibliteki SciKit Learn do stworzenia modułu dla 
języka Python, który umożliwi sterwowanie dowolnymi aplikacjami przy 
użyciu gestów oraz ruchu dłońmi. 

\section{Wykorzystane technologie}
\subsection{OpenCV}
Biblioteka, dzieki której można wykorzystać obraz z kamery oraz wstępnie
przetworzyć obraz, który zostanie wykorzystany przez bibliotekę MediaPipe.

\subsection{MediaPipe}

\quad
Biblioteka MediaPipe o otwartym źródle, udostępnia wieloplatformowe oraz 
konfigurowalne rozwiązania wykrzustujące uczenie maszynowe w dziedzienie 
rozpoznawania, segmentacji oraz klasyfikacji obiektów wizji komputerowej. 
Niektórymi z rozwiązań są:

\begin{itemize}[noitemsep]
    \item Rozpoznawanie twarzy
    \item Segmentacjia włosów oraz twarzy
    \item Rozpoznawnie oraz określanie rozmiarów obiektów trójwymiarowych 
          na podstawie obrazu dwuwymiarowego. 
\end{itemize}

Platformy/Języki programowania obsługiwane przez MediaPipe:

\begin{itemize}[noitemsep]
    \item Android
    \item IOS
    \item JavaScript
    \item Python
    \item C++
    \item Coral
\end{itemize}

????
Pozwoli na rozpoznaie dłoni oraz jej elementów charakterystycznych, 
takich jak nagdarstek, stawy oraz końcówki palców. 
???


\subsection{SciKit Learn}

SciKit Learn to biblioteka, która oferuje różnego typu metody uczenia
maszynowego. Biblioteka zawiera algorytmy klasyfikacji, regresjii oraz 
analizy skupień. 
Biblioteka pozwala na wykorzystanie metod klasyfikacji uczenia maszynowego. 
Co pozwli na rozpoznanie gestów dłoni. 
