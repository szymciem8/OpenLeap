%%%%%%%%%%%%%%%%%%%%%%%%%%%%%%%%%%%%%%%%%%
%                                        %
% Szablon pracy dyplomowej inzynierskiej % 
%                                        %
%%%%%%%%%%%%%%%%%%%%%%%%%%%%%%%%%%%%%%%%%%



\documentclass[a4paper,twoside,12pt]{book}
\usepackage[utf8]{inputenc}                                      
\usepackage[T1]{fontenc}  
\usepackage{amsmath,amsfonts,amssymb,amsthm}
\usepackage[british,polish]{babel} 
\usepackage{indentfirst}
\usepackage{lmodern}
\usepackage{graphicx} 
\usepackage[hidelinks]{hyperref}
\usepackage{booktabs}
%\usepackage{tikz}
%\usepackage{pgfplots}
\usepackage{mathtools}
\usepackage{geometry}
\usepackage{amsmath}
\usepackage[page]{appendix} % toc,
\renewcommand{\appendixtocname}{Dodatki}
\renewcommand{\appendixpagename}{Dodatki}
\renewcommand{\appendixname}{Dodatek}

\usepackage{graphicx}
\usepackage{float}
\graphicspath{{./images/}}

\usepackage{setspace}
\onehalfspacing

\usepackage{dirtree}
\usepackage{hyperref}

\frenchspacing

\usepackage{listings}
%\lstset{
%	language={},
%	basicstyle=\ttfamily,
%	keywordstyle=\lst@ifdisplaystyle\color{blue}\fi,
%	commentstyle=\color{gray}
%}

%%%%%%%%%%%%%%%%%%%%%%%%%%%
% listingi 
\usepackage{listings}
\lstset{%
	language=python,%
	commentstyle=\textit,%
	identifierstyle=\textsf,%
	keywordstyle=\sffamily\bfseries, %\texttt, %
	%captionpos=b,%
	tabsize=3,%
	frame=lines,%
	numbers=left,%
	numberstyle=\tiny,%
	numbersep=5pt,%
	breaklines=true,%
	morekeywords={descriptor_gaussian,descriptor,partition,fcm_possibilistic,dataset,my_exception,exception,std,vector},%
	escapeinside={@*}{*@},%
	%texcl=true, % wylacza tryb verbatim w komentarzach jednolinijkowych
}

\lstset{%
	language=bash,%
	commentstyle=\textit,%
}
%%%%%%%%%%%%%%%%%%%%%%%%%%%%%%%%%%%%


%%%%%%%%%

%%%% TODO LIST GENERATOR %%%%%%%%%

%\usepackage{tikz}
%\usepackage{manfnt}   % dangerous sign 
\usepackage{color}
\definecolor{brickred}      {cmyk}{0   , 0.89, 0.94, 0.28}

\makeatletter \newcommand \kslistofremarks{\section*{Uwagi} \@starttoc{rks}}
  \newcommand\l@uwagas[2]
    {\par\noindent \textbf{#2:} %\parbox{10cm}
{#1}\par} \makeatother


\newcommand{\ksremark}[1]{%
{%\marginpar{\textdbend}
{\color{brickred}{[#1]}}}%
\addcontentsline{rks}{uwagas}{\protect{#1}}%
}

\newcommand{\comma}{\ksremark{przecinek}}
\newcommand{\nocomma}{\ksremark{bez przecinka}}
\newcommand{\styl}{\ksremark{styl}}
\newcommand{\ortografia}{\ksremark{ortografia}}
\newcommand{\fleksja}{\ksremark{fleksja}}
\newcommand{\pauza}{\ksremark{pauza `--', nie dywiz `-'}}
\newcommand{\kolokwializm}{\ksremark{kolokwializm}}

%%%%%%%%%%%%%% END OF TODO LIST GENERATOR %%%%%%%%%%%

%%%%%%%%%%%% ZYWA PAGINA %%%%%%%%%%%%%%%
% brak kapitalizacji zywej paginy
\usepackage{fancyhdr}
\pagestyle{fancy}
\fancyhf{}
\fancyhead[LO]{\nouppercase{\it\rightmark}}
\fancyhead[RE]{\nouppercase{\it\leftmark}}
\fancyhead[LE,RO]{\it\thepage}


\fancypagestyle{tylkoNumeryStron}{%
   \fancyhf{} 
   \fancyhead[LE,RO]{\it\thepage}
}

\fancypagestyle{NumeryStronNazwyRozdzialow}{%
   \fancyhf{} 
   \fancyhead[LO]{\nouppercase{\it\rightmark}}
   \fancyhead[RE]{\nouppercase{\it\leftmark}}
   \fancyhead[LE,RO]{\it\thepage}
}


%%%%%%%%%%%%% OBCE WTRETY  
\newcommand{\obcy}[1]{\emph{#1}}
\newcommand{\ang}[1]{{\selectlanguage{british}\obcy{#1}}}
%%%%%%%%%%%%%%%%%%%%%%%%%%%%%

% polskie oznaczenia funkcji matematycznych
\renewcommand{\tan}{\operatorname {tg}}
\renewcommand{\log}{\operatorname {lg}}

% jeszcze jakies drobiazgi

\newcounter{stronyPozaNumeracja}

\newcommand{\hcancel}[1]{%
    \tikz[baseline=(tocancel.base)]{
        \node[inner sep=0pt,outer sep=0pt] (tocancel) {#1};
        \draw[red] (tocancel.south west) -- (tocancel.north east);
    }%
}%

\newcommand{\miesiac}{%
  \ifcase\the\month
  \or styczeń% 1
  \or luty% 2
  \or marzec% 3
  \or kwiecień% 4
  \or maj% 5
  \or czerwiec% 6
  \or lipiec% 7
  \or sierpień% 8
  \or wrzesień% 9
  \or październik% 10
  \or listopad% 11
  \or grudzień% 12
  \fi}


%%%%%%%%%%%%%%%%%%%%%%%%%%%%%%%%%%%%%%%%%%%%%%
% Helvetica font macros for the title page:
\newcommand{\headerfont}{\fontfamily{phv}\fontsize{18}{18}\bfseries\scshape\selectfont}
\newcommand{\titlefont}{\fontfamily{phv}\fontsize{18}{18}\selectfont}
\newcommand{\otherfont}{\fontfamily{phv}\fontsize{14}{14}\selectfont}

\newcommand{\autor}{Szymon Ciemała}
\newcommand{\promotor}{dr inż. Krzysztof Jaskot}
\newcommand{\konsultant}{dr inż. Imię Nazwisko}
\newcommand{\tytul}{Tytuł pracy dyplomowej inżynierskiej}
\newcommand{\polsl}{Politechnika Śląska}
\newcommand{\wydzial}{Wydział Automatyki, Elektroniki i Informatyki}
\newcommand{\kierunek}{Kierunek: Automatyka i Robotyka}


\begin{document}
%\kslistofremarks 
%%%%%%%%%%%%%%%%%%%%%%%%%%%%%%%%%%%%%%%%%%%%%%%%%%%%%%%
%%%%%%%%%%%%%%%%%%  STRONA TYTULOWA %%%%%%%%%%%%%%%%%%%
%%%%%%%%%%%%%%%%%%%%%%%%%%%%%%%%%%%%%%%%%%%%%%%%%%%%%%%
\pagestyle{empty}
{
	\newgeometry{top=2.5cm,%
	             bottom=2.5cm,%
	             left=3cm,
	             right=2.5cm}
	\sffamily
	\rule{0cm}{0cm}
	
	\begin{center}
	\includegraphics[width=29mm]{logo_pl.jpg}
	\end{center} 
	\vspace{1cm}
	\begin{center}
	\headerfont \polsl
	\end{center}
	\begin{center}
	\headerfont \wydzial
	\end{center}
	\begin{center}
	\headerfont \kierunek
	\end{center}
	\vfill
	\begin{center}
	\titlefont Praca dyplomowa inżynierska
	\end{center}
	\vfill
	
	\begin{center}
	\otherfont \tytul\par
	\end{center}
	
	\vfill
	
	\vfill
	 
	\noindent\vbox
	{
		\hbox{\otherfont autor: \autor}
		\vspace{12pt}
		\hbox{\otherfont kierujący pracą: \promotor}
		\vspace{12pt}  % zakomentuj, jezeli nie ma konsultanta
		\hbox{\otherfont konsultant: \konsultant} % zakomentuj, jezeli nie ma konsultanta
	}
	\vfill 
 
   \begin{center}
   \otherfont Gliwice,  \miesiac\ \the\year
   \end{center}	
	\restoregeometry
}
  

%\cleardoublepage
 

\rmfamily
\normalfont

%%%%%%%%%%%%%%%%%%%%%%%%%%%%%%%%%%%%%%%%%%%%%%%%%%%%%%%
%%%%%%%%%%%%%%%%%%%% SPIS TRESCI %%%%%%%%%%%%%%%%%%%%%%
%%%%%%%%%%%%%%%%%%%%%%%%%%%%%%%%%%%%%%%%%%%%%%%%%%%%%%%
\pagenumbering{Roman}
\pagestyle{tylkoNumeryStron}
\tableofcontents

%%%%%%%%%%%%%%%%%%%%%%%%%%%%%%%%%%%%%%%%%%%%%%%%%%%%%
\setcounter{stronyPozaNumeracja}{\value{page}}
\mainmatter
\pagestyle{empty}

%%%%%%%%%%%%%%%%%%%%%%%%%%%%%%%%%%%%%%%%%%%%%%%%%%%%%%%
%%%%%%%%%%%%%%%%%%% Streszczenie %%%%%%%%%%%%%%%%%%%%%%
%%%%%%%%%%%%%%%%%%%%%%%%%%%%%%%%%%%%%%%%%%%%%%%%%%%%%%%

\chapter{Streszczenie}

% Streszczenie pracy -odpowiednie pole w systemie APD powinno zawierac kopie tego streszczenia. Streszczenie, wraz ze slowami kluczowymi, nie powinno przekroczyc jednej strony.
%{\bf Slowa kluczowe:} 2-5 slow (fraz) kluczowych, oddzielonych przecinkami

%%Faktyczne streszczenie

\quad Praca inżynierska "Rozpoznawanie obiektów z wykorzystaniem biblioteki OpenCV", łączy tematykę wizji komputerowej oraz algorytmów uczenia maszynowego. 

\quad Praca skupia się na stworzeniu wygodnej w użyciu oraz powszechnie dostępnej biblioteki umożliwiającej wykorzystanie gestów oraz pozycji dłoni w dowolnych projektach napisanych w języku Python. 

\quad Do napisania pracy wykorzystano biblioteki języka Python o otwartym kodzie źródłowym, głównie OpenCV, MediaPipe oraz SciKit Learn. 

\addcontentsline{toc}{chapter}{Streszczenie}

%\cleardoublepage

\pagestyle{NumeryStronNazwyRozdzialow}

%%%%%%%%%%%%%% wlasciwa tresc pracy %%%%%%%%%%%%%%%%%

%%%%%%%%%%%%%%%%%%%%%% Wstęp %%%%%%%%%%%%%%%%%%%%%%%%

\chapter{Wstęp}
\begin{itemize}
\item wprowadzenie w problem/zagadnienie
\item osadzenie problemu w dziedzinie
\item cel pracy
\item zakres pracy
\item zwięzła charakterystyka rozdziałów
\item jednoznaczne określenie wkładu autora, w przypadku prac wieloosobowych – tabela z autorstwem poszczególnych elementów pracy
\end{itemize}

\newpage

\section{Wprowawdzenie w problem}
\quad Rozwój technolgii w ostatnich czasach przyczynił się do coraz częstszego wykorzystywania wizji komputerowej oraz metod uczenia maszynowego do rozpoznawania oraz klasyfikacji różnego typu obiektów, w tym części ludzkiego ciała. Pozwala to na interakcję człowieka z aplikacjami, często w spób bardziej naturalny. 

\section{Cel pracy}
Projekt inżynierski ma na celu stworzenie biblioteki w języku Python, która pozwoli na przystępne wykorzystanie algorytmów rozpoznawania gestów oraz ruchu dłoni. Biblioteka powinna oferować gotowe rozwiązania, na przykład przygotowane modele matematyczne pozwalające na rozpoznawanie języka migowego oraz gestów podstawowych. Dodatkowo powinna pozowlić na wyznaczenie pozycji dłoni, jej typu oraz wartości charakterystycznych, na przykład odlgłości między końcówkami wybranych palców czy kąta obrotu dłoni. 

\section{Osadzenie problemu w dziedzienie}
\quad Aktualnie istnieją częściowo gotowe rozwiązania pozwalające na rozpoznanie i klasyfikację dłoni - bibliotka MediaPipe. 

\section {Charakterystyka rozdziałów}

% \section{Ogólnie}

% Praca przedstawia wykorzystanie bibliotek OpenCV, MediaPipe oraz metod 
% uczenia maszynowego bibliteki SciKit Learn do stworzenia modułu dla 
% języka Python, który umożliwi sterwowanie dowolnymi aplikacjami przy 
% użyciu gestów oraz ruchu dłońmi. 

% \section{Wykorzystane technologie}
% \subsection{OpenCV}
% Biblioteka, dzieki której można wykorzystać obraz z kamery oraz wstępnie
% przetworzyć obraz, który zostanie wykorzystany przez bibliotekę MediaPipe.

% \subsection{MediaPipe}

% \quad
% Biblioteka MediaPipe o otwartym źródle, udostępnia wieloplatformowe oraz 
% konfigurowalne rozwiązania wykrzustujące uczenie maszynowe w dziedzienie 
% rozpoznawania, segmentacji oraz klasyfikacji obiektów wizji komputerowej. 
% Niektórymi z rozwiązań są:

% \begin{itemize}
%     \item Rozpoznawanie twarzy
%     \item Segmentacjia włosów oraz twarzy
%     \item Rozpoznawnie oraz określanie rozmiarów obiektów trójwymiarowych na podstawie obrazu dwuwymiarowego. 
% \end{itemize}

% Platformy/Języki programowania obsługiwane przez MediaPipe:

% \begin{itemize}
%     \item Android
%     \item IOS
%     \item JavaScript
%     \item Python
%     \item C++
%     \item Coral
% \end{itemize}

% ????
% Pozwoli na rozpoznaie dłoni oraz jej elementów charakterystycznych, 
% takich jak nagdarstek, stawy oraz końcówki palców. 
% ???


% \subsection{SciKit Learn}

% SciKit Learn to biblioteka, która oferuje różnego typu metody uczenia
% maszynowego. Biblioteka zawiera algorytmy klasyfikacji, regresjii oraz 
% analizy skupień. 
% Biblioteka pozwala na wykorzystanie metod klasyfikacji uczenia maszynowego. 
% Co pozwli na rozpoznanie gestów dłoni. 


%%%%%%%%%%%%%%%%%%%%% Analiza %%%%%%%%%%%%%%%%%%%%%%%

\chapter{[Analiza tematu]}

\begin{itemize}
\item sformułowanie problemu
\item osadzenie tematu w kontekście aktualnego stanu wiedzy ( state of the art ) o poruszanym problemie
\item  studia literaturowe \cite{bib:artykul,bib:ksiazka,bib:konferencja,bib:url} -  opis znanych rozwiązań (także opisanych naukowo, jeżeli problem jest poruszany w publikacjach naukowych), algorytmów, 
\end{itemize}

\section {Sformułowanie problemu}
\quad Cele projektowe można podzielić na trzy części. 

\begin{itemize}
    \item Wyznaczenie pozyci dłoni, odległości pomiędzy wybranymi palcami oraz jej kąta obrotu. 
    \item Rozpoznawanie gestów dłoni w dwóch trybach: prostym (parę dostępnych gestów) oraz zaawansowanym, który rozoznaje alfabet języka migowego. 
    \item Dostępność biblioteki poprzez platfromę PyPi. 
\end{itemize}

\quad Stworzenie modułu będzie wymagało napisania klasy wykorzystującej odpowiednie biblioteki pozwalających na rozpoznanie elementów charakterystycznych dłoni oraz przetworzenia obrazu. Obraz będzie pochodził z kamerki internetowej, który zostanie odpowiednio przetworzony z wykorzystaniem funkcji dostępnych poprzez bibliotekę OpenCV. Przetworzony obraz zostanie wykorzystany przez metody biblioteki MediaPipe, która pozowli na rozpoznanie elementów charakterystycznych dłoni. W napisanej klasie zostaną zaimplementowane metody, które pozowlą na rozpoznanie typu dłoni (prawa, lewa), odległości między paliczkiem palca wskazującego oraz kciuka, oraz obrotu dłoni. 

\quad Kolejnym elementem jest wygenerowanie modelu matematycznych klasyfikujących gesty dłoni. W tym celu zostanie wykorzystana bibliotegk SciKit Learn wraz z dostępnymi poprzez nią algorytmamy ucznia maszynowego. Odpowiednio zebrane dane pozowlą na przeprowadzeni procesu ucznie dla paru wybranych algorytmów. 

\quad Najważniejszym elementem pracy jest powyżej wymieniony ostatni punkt listy. Dzięki wykorzystaniu platformy PyPi deweloperzy będą mogli przy pomocy programu \textbf{pip} za pomocą jednej komendy zainstalować paczkę wraz ze wszystkimi wymaganymi zależnościami. 

% \chapter{Założenia projektowe}

% \section{Budowa modułu}
% Moduł został napisany przy pomocy paradygmatu programowania
% obiektowego, co pozwala na przystępne wykorzystanie biblioteki w dowonych 
% projektach wymagających osbługi gestów. 

% \section{Dostęp}
% Całość projektu będzie dostępna na platformie GitHub wraz z możliwością 
% pobrania przy pomocy programu pip ze zdlanego repozytorium PyPi, dzięki czemu 
% pozowli to na sprawne i proste wykorzystanie modułu w dowolnym 
% projekcie. 

%%%%%%%%%%%%%%% Wymagania i Narzędia %%%%%%%%%%%%%%%%

% \chapter{Część techniczna/praktyczna}

\chapter{Wymagania i narzędzia}

\begin{itemize}
\item wymagania funkcjonalne i niefunkcjonalne
\item przypadki użycia (diagramy UML) - dla prac, w których mają zastosowanie
\item opis narzędzi, metod eksperymentalnych, metod modelowania itp.
\item metodyka pracy nad projektowaniem i implementacją - dla prac, w których ma to zastosowanie
\end{itemize}


\section{Wymagania}

\quad Klasa powinna zawierać w sobie wszystkie niezbędne funkcje oraz parametry pozwalające na wykorzysatnie jej w dowolnym projekcie. Modele rozpoznające gesty powinny zostać uprzednio przygotowane. 

\quad Pobranie modułu powinno być możliwe poprzez wykorzystanie standardowego programu do zarządzania paczkami \textbf{pip}. Ten menedżer pozwala na instalacje paczki wraz z jej zależnościami, czyli innymi paczkami wymaganami do porwanego działania pobieranego modułu. 

\section{Narzędzia}

\subsection{System operacyjny Ubuntu}
\quad Do stworzenia oprogramowania została wybrana dystrybucja Ubuntu systemu Linux. Dzięki takim elementom jak menedżer pakietów lub wbudowane wspracie dla języka Python.

\subsection{Python}
\quad Do napisanie biblioteki został wykorzystany język skryptowy Python. Ze względu na swoją budowę, Python nie należy do najbardziej wydajnych języków. Python jest językiem interpretowalnym, co oznacza, że jego program nie zostaje przekompilowany do kodu maszynowego, a zostaje on zinterpretowany przez program nazywany interpreterem. 
\quad Pomimo niskiej wydajności języka Python, jest on dobrym narzędziem do tworzenia aplikacji wykorzystujących wizję komputerową oraz uczenia maszynowe. Głównie ze względu na dostępność oraz jakość bibliotek o wolnym źródle.  

\subsection{iPython}
\quad To iteraktywna powłoka dla języka Python, która rozszerza jego działanie o interospekcję, czyli możliwość wykonywania poprzednich części programu zapisanych w komórkach. Dodatkowo IPython oferuje dodatkową składnię powołoki oraz wykonywanie komend wiersza poleceń. 

\subsection{Jupyter Notebook}
\quad Dodatkowym narzędziem jest Jupyter Notebook, który pozwala na wykonwyanie części programu w osobnych komórkach. Pozwala to na łatwiejsze odnajdowanie błędów w programie.
\quad Jupyter Notebook został wykorzystany do napisania programu trenującego modele uczenia mszynowego rozponzące gesty. Dzięki strukturze takiego pliku, można w prosty sposób wykonywać program w danej komórce wiele razy. Na przykład pobieranie danych, dla różnych gestów. 

\subsection{Visual Studio Code - Środowisko Pracy}
\quad Całość klasy została napisana z wykorzystaniem edytora Visual Studio Code. Ten edytor został wybrany ze względu na jego wielozadaniowość. Visual Studio Code obsługuje jednocześnie programy napisane w Pythonie oraz pliki z rozszerzeniem \textbf{.ipynb}, czyli tak zwane zeszyty, które wykorzystują interaktywnego Pythona, czyli język iPython. 

\subsection{Platforma PyPi}
\quad Platforma PyPi jest platformą, na której udostępniane są modułu napisane w języku Python. Aktualnie jest ona zarządzana poprzez fundację "Python Software Foundation". Dzięki tej platfomie, zupełnie za darmo można pobierać znajdujące się na niej oprogramowanie oraz udostępniać własne. Program \textbf{pip} w głónwej mierze korzysta z PyPi jako głównego repozytorium z paczkami. 

\subsection{Programy bdist\_wheel oraz sdist}
\quad Oba programy są niezbędne do przygotowania w pełni działającej paczki udostępnionej przez platforme PyPi. Program \textbf{sdist} pozwala na stworzenie źródłowej dystrubucji, czyli w praktce pliku typu \textbf{.zip}, \textbf{.tar} czy też \textbf{.rar} oraz wielu innych, w których znajdują się wybrane pliki. 
\quad Kolejnym programem jest \textbf{bdist\_wheel}, który odpowiada za stworzenie uprzednio paczki typu \textbf{WHEEL}, która w przypadku wykorzystania plików napisanych w języku \textbf{C} uprzednio je kompiluje, dzięki czemu nie jest wymagany kompilator po stronie użytkownika. Ostatecznie, \textbf{WHEEL} pozwala na szybszą instalację paczki w porównaniu z budowaniem jej z plików źródłowych. 



%%%%%%%%%%%%%%%%%%%%%%%%%%%%%%%%%%%%%%%%%%%%%%%%%%%%%%%%%%%%%%%%%%%%
%%%%%%%%%%%%%%%%%%%%%%%%%%%%%% OPEN CV %%%%%%%%%%%%%%%%%%%%%%%%%%%%%
%%%%%%%%%%%%%%%%%%%%%%%%%%%%%%%%%%%%%%%%%%%%%%%%%%%%%%%%%%%%%%%%%%%%
\subsection{OpenCV - przygotowanie obrazu z kamery}


%%%%%%%%%%%%%%%%%%%%%%%%%%%%%%%%%%%%%%%%%%%%%%%%%%%%%%%%%%%%%%%%%%%%
%%%%%%%%%%%%%%%%%%%%%%%%%%%% MEDIA PIPE %%%%%%%%%%%%%%%%%%%%%%%%%%%%
%%%%%%%%%%%%%%%%%%%%%%%%%%%%%%%%%%%%%%%%%%%%%%%%%%%%%%%%%%%%%%%%%%%%

\subsection{MediaPipe - Elementy charakterystyczne}

\quad Biblioteka MediaPipe o otwartym źródle, udostępnia wieloplatformowe oraz konfigurowalne rozwiązania wykrzustujące uczenie maszynowe w dziedzienie rozpoznawania, segmentacji oraz klasyfikacji obiektów wizji komputerowej. Niektórymi z rozwiązań są:

\begin{itemize}
    \item Rozpoznawanie twarzy
    \item Segmentacjia włosów oraz twarzy
    \item Rozpoznawnie oraz określanie rozmiarów obiektów trójwymiarowych 
          na podstawie obrazu dwuwymiarowego. 
\end{itemize}

\quad Model modułu MediaPipe pozwala na wyznaczenie pozycji 21 elementów charakterystycznych dłoni. Współrzędne X i Y są znomralizowane względem rozdzielczości obrazu kamery. Współrzędna X względem liczby pikseli w osi X, a współrzędna Y względem liczby pikseli w osi Y. Oś Z jest prostopadła do osi X i Y, z punktem początkowym w punkcie określającym pozycję nadgarstka. Współrzędna Z jest znormalizowana względem szerokości obrazu kamery, tak jak współrzędna X. 


\quad Pozycje nadgarstka, paliczków oraz stawów dłoni zostaną wykorzystane do obliczenia obrotu dłoni względem punkut 0 oraz do wytrenowania modeli uczenia maszynowego, których zadaniem będzie rozpoznawnie wybranych gestów. 

\subsection{Generowanie grafiki dłoni}

\quad Generowanie grafiki nałożonej na daną dłoń wykonuje się przy pomocy przygotowanej funkcji biblioteki MediaPipe, która współpracuje z OpenCV. 

%%%%%%%%%%%%%%%%%%%%%%%%%%%%%%%%%%%%%%%%%%%%%%%%%%%%%%%%%%%%%%%%%%%%
%%%%%%%%%%%%%%%%%%%%%%%%%%% SciKit Learn %%%%%%%%%%%%%%%%%%%%%%%%%%%
%%%%%%%%%%%%%%%%%%%%%%%%%%%%%%%%%%%%%%%%%%%%%%%%%%%%%%%%%%%%%%%%%%%%


\section{SciKit Learn - uczenie maszynowe}

\quad SciKit Learn to biblioteka, która oferuje różnego typu metody uczenia maszynowego. Biblioteka zawiera algorytmy klasyfikacji, regresjii oraz analizy skupień. Przykładowym algorytmami w bibliotece są:

\begin{itemize}
    \item Las losowy - polegająca na konstruowaniu wielu drzew decyzyjnych w czasie uczenia. 
    \item Algorytm centroidów - algorytm wykorzystywany w analizie skupień.
    \item Maszyna wektorów nośnoych - algorytm klasyfikujący, często wykorzystywany w procesie rozpoznawania obrazów. 
\end{itemize}

O wszystkich dostępnych algorytmach informacje można znaleźć w ogólnodostępnej dokumentacji biblioteki. 


%%%%%%%%%%%%%% Specyfikacja Zewnętrzna %%%%%%%%%%%%%%

% \chapter{Część techniczna/praktyczna}

% \chapter{Wymagania i narzędzia}

\chapter{[Właściwy dla kierunku -- np. Specyfikacja zewnętrzna]}
% Jeśli to Specyfikacja zewnętrzna:
% \begin{itemize}
% \item  wymagania sprzętowe i programowe
% \item  sposób instalacji
% \item  sposób aktywacji
% \item  kategorie użytkowników
% \item  sposób obsługi
% \item   administracja systemem
% \item  kwestie bezpieczeństwa
% \item  przykład działania
% \item  scenariusze korzystania z systemu (ilustrowane zrzutami z ekranu lub generowanymi dokumentami)
% \end{itemize}

% \section{Funkcjonalność modułu}
% \quad Paczkę można wykorzystać tam, gdzie wymagane jest wykorzystanie gestów oraz ruchów dłoni. Paczka może znnaaleźć zastosowanie aplikacjach użytku codziennego, robotyce lub pełnić formę peryferium komputerowego.  

\section{Wymagania sprzętowe}
\subsection{Kamera}
\quad Elementem niezbędnym do korzystania z modułu OpenLeap jest kamera, która pozwoli na pozyskanie obrazu. Rozdzielczość matrycy kamery powinna być wystarczająco duża, aby pozwolić na rozpoznanie dłoni. Nie ma tutaj minimalnych wymagań, większa rozdzielczość, czy też lepsza praca w warunkach niskiego oświetlenia kamery pozwoli na poprawniejsze działania algorytmów identyfikujących dłoń. Podobnie ma się liczba klatek na sekundę, która powinna być wystarczająco wysoka, tak aby można było sprawnie korzystać z możliwości oprogramowania. 

\subsection{Komputer}
\quad Jednostka obliczeniowa powinna zostać wyposażona w system operacyjny dający możliwość obsługi języka Python, taki warunek spełnia większość systemów operacyjnych na rynku. Komputer powinien spełniać minimalne wymagania w kwestii wydajności przetwarzania obrazu. Fakt istnienia możliwości instalacji i wykorzystania biblioteki MediaPipe przez minikomputery Raspberry Pi w wersji 3 i 4 oznacza, że wymagania nie są wysokie. 


\section{Instalacja paczki}
\quad Instalacja paczki odbywa się poprzez wykorzystanie programu \textbf{pip}, który jest instalowany automatycznie razem z językiem Python. W zależności od wybranego systemu operacyjnego komenda może przybierać różne formy, ogólnie można przyjąć poniższy zapis \ref{lst:installcom}. Komenda powinna zostać wykonane poprzez powłokę \textbf{bash} lub inną dostępną w systemie Unix-owym lub poprzez wiersz poleceń w systemie Windows. 

\begin{lstlisting}[language=bash, style=command, label={lst:installcom}, caption={Instalacja paczki}]
    $ pip install openleap
\end{lstlisting}

\quad Informacje na temat biblioteki można znaleźć na platformie PyPi pod linkiem: \textbf{\href{https://pypi.org/project/openleap/}{https://pypi.org/project/openleap/}}. Na tej stronie znajduje się opis modułu, instrukcja instalacji oraz przykładowe programy i możliwości wykorzystania.

\begin{figure}[H]
    \begin{center}
        \includegraphics[width=15cm]{../images/pypi_page.png}
        \caption{Strona modułu OpenLeap na PyPi}
    \end{center}
\end{figure}

\section{Program testowy}

\quad Paczkę można przetestować korzystając z dostępnych metod klasy. W ramach testu istnieje możliwość napisania programu wyłącznie w powłoce języka Python. W pierwszym kroku programu zostaje zaimportowana paczka \textbf{openleap}. Zostaje stworzony obiekt kontrolera z wybranymi parametrami, dokładny opis parametrów znajduje się w późniejszej części rozdziału, w podsekcji (\ref{parametry}). Metoda \textbf{loop()} odpowiada za wywołanie głównej logiki programu odpowiedzialnej za generowanie danych oraz wyświetlanie ich w wybranym miejscu (powłoka, okno graficzne). 

\begin{lstlisting}[language=python, style=programming, label={lst:simple_program}, caption={Program testowy}]
    import openleap

    controller = openleap.OpenLeap(screen_show=True, 
                                   screeen_type='BLACK', 
                                   show_data_on_image=True, 
                                   gesture_model='basic')
    
    controller.loop()
\end{lstlisting}

\quad Przykładowy program pozwoli na wyświetlenie okna z widocznymi dłońmi wraz z oznaczonymi punktami charakterystycznymi oraz opisem parametrów, takich jak obrót dłoni względem nadgarstka, jej pozycja względem lewego górnego rogu obrazu kamery czy rozpoznany gest. Zrzut ekranu testowego programu znajduje się poniżej na rysunku \ref{fig:prog_screen_1}. 

\begin{figure}[H]
    \begin{center}
        \includegraphics[width=9cm]{../images/example_program.png}
        \caption{Zrzut ekranu programu testowego}
        \label{fig:prog_screen_1}
    \end{center}
\end{figure}

\quad Dane zostają zapisane w słowniku składającym się z obiektów typu \textbf{dataclass}, które przechowują wygenerowane informacje o danej dłoni. Operację pobrania danych można wykonać w poniżej przedstawiony sposób. Klasa \textbf{dataclass} zostanie dokładniej opisana w kolejnym rozdziale.

\begin{lstlisting}[language=python, style=programming, caption={Odczyt gestów}]
    if controller.data['right'].gesture == 'open':
        print('Right hand is opened!')
    elif controller.data['right'].gesture == 'fist':
        print('Right hand is closed!')
\end{lstlisting}

\quad Program oferuje możliwość odczytu odległości między końcem palca wskazującego a końcem kciuka. Te punkty zostały oznaczone poprzez niebieską linię oraz wartość \textbf{distance}, co można zobaczyć na rysunku \ref{fig:prog_screen_1}. Przykład odczytu tej wartości znajduje się w poniższym przykładzie. 

\begin{lstlisting}[language=python, style=programming, caption={Odczyt edległości między palcami}]
    if controller.data['right'].distance < 20:
        print('Click has been detected!')
\end{lstlisting}

\subsection{Parametry}
\label{parametry}
\quad Obiekt kontrolera może zostać stworzony z wybranymi parametrami. Można wybrać czy należy wyświetlić podgląd kamery, czy wyświetlić dane na podglądzie oraz w konsoli, oraz wybrać tło podglądu, czarne lub obraz z kamery. 

\quad Obiekt klasy openleap przyjmuje do inicjalizatora parametry, określające czy należy wyświetlić okno graficzne, dane lub który model rozpoznający gesty wybrać. Wszystkie atrybuty klasy zostaną dokładnie opisane w kolejnym rozdziale. 

\begin{enumerate}
    \item \textbf{screen\_show} - podgląd okna, typ \textbf{boolean}
    \item \textbf{screen\_type} - typ tła
    \begin{itemize}
        \item \enquote{CAM} - obraz z kamery
        \item \enquote{BLACK} - czarne tło
    \end{itemize}
    \item \textbf{show\_data\_in\_console} - wyświetlanie danych w konsoli, typ \textbf{boolean}
    \item \textbf{show\_data\_on\_image} - wyświetlanie danych w oknie graficznym, typ \textbf{boolean}
    \item \textbf{normalized\_position} - wyświetlanie znormalizowanej pozycji, typ \textbf{boolean}
    \item \textbf{model} - wybór modelu rozpoznającego gesty
    \begin{itemize}
        \item \enquote{basic} - gesty podstawowe
        \item \enquote{sign\_language} - język migowy
    \end{itemize}
\end{enumerate}


\begin{figure}[H]
    \centering
    \subfloat[Obraz z kamery.]{\includegraphics[width=7.3cm]{../images/example_program.png}\label{fig:f1}}
    \hfill
    \subfloat[Czarne tło.]{\includegraphics[width=7.3cm]{../images/cam_example.png}\label{fig:f2}}
    \caption{Parametry wyświetlanego obrazu.}
\end{figure}

\subsection{Dostępne funkcje}

\subsection{Główny wątek}
\quad Działanie kontrolera opiera się na wywoływaniu funkcji \textbf{main()}. Tą funkcję można wywołać za pomocą wybudowanej funkcji \textbf{loop()}, która po prostu wywołuje funkcję \textbf{main()} w pętli wraz z warunkiem wyjścia z programu. Takie podejście może ułatwić stosowanie kontrolera w osobnym wątku aplikacji. 
\quad Alternatywnym podejściem jest po prostu zastosowanie funkcji \textbf{main()} w pętli budowanego programu. Takie rozwiązanie daje podobny rezultat, ale bez konieczności wykorzystania osobnego wątku. 

\section{Przykłady użycia}
\subsection{Rozpoznawanie alfabetu w języku migowym}
\quad Pierwszym przykładem zastosowania jest wykorzystanie paczki do rozpoznawnia alfabetu języka migowego. Taki program może umożliwić komunikację między osobą głuchoniemą posługującą się językiem migowym, a osobą, które takiego jęzka nie zna. 


\begin{figure}[H]
    \begin{center}
        \includegraphics[width=15cm]{../images/american_sign_language.jpg}
        \caption{Gesty alfabetu języka migowego}
    \end{center}
\end{figure}

\quad Oprócz modelu rozpoznającego gesty alfabetu języka migowego, klasa została wyposażona w model rozpoznający podstawowe gesty, czyli gesty otwartej i zamkniętej dłoni. Wybór odpowiedniego modelu odbywa się w momencie tworzenia obiektu poprzez ustawienie parametru \textbf{gesture\_model}.


\begin{figure}[H]
    \centering
    \subfloat[Litery A i B.]{\includegraphics[width=7.3cm]{../images/a_b.png}\label{fig:f1}}
    \hfill
    \subfloat[Litery C i X.]{\includegraphics[width=7.3cm]{../images/c_x.png}\label{fig:f2}}
    \hfill
    \subfloat[Litery F i O.]{\includegraphics[width=7.3cm]{../images/f_o.png}\label{fig:f3}}
    \caption{Rozpoznawanie języka migowego .}
\end{figure}

\subsection{Interaktywny Kiosk}
\quad Kolejnym przykładem jest interktywny kiosk, który pozwala na złożenie zamówienia w sposób, który nie wymaga dotykania ekranu dotykowego. W dobie pandemii takie rozwiązanie może potencjalnie przyczynić się do spowolnienia rozprzestrzeniania się różnego rodzaju wirusów i drobnoustrojów. 


\begin{figure}[H]
    \begin{center}
        \includegraphics[width=15cm]{../images/checkout_window.png}
        \caption{Zrzut ekranu kiosku interaktywnego}
    \end{center}
\end{figure}

\quad Wskaźnik kiosku interaktywnego jest sterowany poprzez pozycję dłoni. Kliknięcie w przycisk zostaje aktywowane za pomocą wykrycia odpowiednio małej odlęgłości między końcówką palca wskazującego, a końcówką kciuka. 

\quad Odległość między tymi dwoma punktami jest obliczana automatycznie poprzez wbudowaną funkcję klasy OpenLeap. 

\subsection{Dobór koloru}
\quad Ostatnim przykładem jest wykorzystanie dłoni jako kontrolera, za którego pomocą można wybrać dowolny kolor. Takie zastosowanie może zostać wykorzystane w pracy grafika komputerowego. Dzięki temu użytkownik będzie mógł zmieniać kolor wykorzysytywanego narzędzia bez przerywania pracy, na przykład malowania. Kolor można ustawiać tyko wtedy kiedy gest lewej ręki jest gestem otwartej dłoni. 


\begin{figure}[H]
    \centering
    \subfloat[Widok dłoni.]{\includegraphics[width=7.3cm]{../images/yellow_hand_deg.png}\label{fig:f1}}
    \hfill
    \subfloat[Dobrany kolor.]{\includegraphics[width=7.3cm]{../images/yellow.png}\label{fig:f2}}
    \caption{Obrót dłoni o 30 stopni.}
\end{figure}

\begin{figure}[H]
    \centering
    \subfloat[Widok dłoni.]{\includegraphics[width=7.3cm]{../images/blueish_hand_deg.png}\label{fig:f1}}
    \hfill
    \subfloat[Dobrany kolor.]{\includegraphics[width=7.3cm]{../images/blueish.png}\label{fig:f2}}
    \caption{Obrót dłoni o 96 stopni.}
\end{figure}

\begin{figure}[H]
    \centering
    \subfloat[Widok dłoni.]{\includegraphics[width=7.3cm]{../images/pink_hand_deg.png}\label{fig:f1}}
    \hfill
    \subfloat[Dobrany kolor.]{\includegraphics[width=7.3cm]{../images/pink.png}\label{fig:f2}}
    \caption{Obrót dłoni o 174 stopni.}
\end{figure}

\quad Dodatkową opcją jest ustawienie saturacji poprzez obliczenie jej wartości na podstawie odległości między palcem wskazującym, a kciukiem. 

\begin{figure}[H]
    \centering
    \subfloat[Widok dłoni.]{\includegraphics[width=7.3cm]{../images/zero_sat_hand_deg.png}\label{fig:f1}}
    \hfill
    \subfloat[Dobrany kolor.]{\includegraphics[width=7.3cm]{../images/zero_sat.png}\label{fig:f2}}
    \caption{Ustawienie saturacji}
\end{figure}


\quad Podobnie jak obliczenie odległości między palcami, biblioteka oblicza obrót dłoni wokół osi Z punktu opisującego pozycję nadgarstka.


\section{Stworzenie nowego modelu}

\quad Użtkownik biblioteki może stworzyć własny model rozpoznający gesty. Wykorzystać do tego można program napisany przy pomocy Jupyter Notebook. Plik jest dostępny na platformie \href{https://github.com/szymciem8/OpenLeap/blob/main/Jupyter%20Notebook/Create%20your%20own%20model/Gesture%20Detection.ipynb}{\textbf{GitHub}}. Plik posiada rozszerzenie \textbf{.ipynb}.

\quad Dokładny opis tworzenia modelu zostanie opisany w kolejnym rozdziale dotyczącym specyfikacji wewnętrznej. 

%%%%%%%%%%%%%% Specyfikacja Wewnętrzna %%%%%%%%%%%%%%

\chapter{[Właściwy dla kierunku - np.Specyfikacja wewnętrzna]}
Jeśli to Specyfikacja wewnętrzna:
\begin{itemize}
\item przedstawienie idei
\item architektura systemu
\item opis struktur danych (i organizacji baz danych)
\item komponenty, moduły, biblioteki, przegląd ważniejszych klas (jeśli występują)
\item przegląd ważniejszych algorytmów (jeśli występują)
\item szczegóły implementacji wybranych fragmentów, zastosowane wzorce projektowe
\item diagramy UML
\end{itemize}


\begin{itemize}
    \item wymagania funkcjonalne i niefunkcjonalne
    \item przypadki użycia (diagramy UML) - dla prac, w których mają zastosowanie
    \item opis narzędzi, metod eksperymentalnych, metod modelowania itp.
    \item metodyka pracy nad projektowaniem i implementacją - dla prac, w których ma to zastosowanie
    \end{itemize}
    
    \section{Rozpoznawanie dłoni}
    
    \quad Pierwszym elementem projektu jest rozpoznanie dłoni poprzez wyznacznie pozycji elementów charakterystycznych. Pozycja każdego z tych elementów jest względna wedłgu pozycji nadgarstka. ????

    \subsection{Interpetacja obrazu - schemat UML}
    
    %%%%%%%%%%%%%%%%%%%%%%%%%%%%%%%%%%%%%%%%%%%%%%%%%%%%%%%%%%%%%%%%%%%%
    %%%%%%%%%%%%%%%%%%%%%%%%%%%%%% OPEN CV %%%%%%%%%%%%%%%%%%%%%%%%%%%%%
    %%%%%%%%%%%%%%%%%%%%%%%%%%%%%%%%%%%%%%%%%%%%%%%%%%%%%%%%%%%%%%%%%%%%
    \subsection{OpenCV - przygotowanie obrazu z kamery}
    
    \quad Poprawne działanie modelu MediaPipe wymaga odpowiedniego przygotowania obrazu kamery. Działanie kontrolerolera odbywa się poprzez główną metodę \textbf{main()}.
    
    Metoda \textbf{main()} jest główną funkcją, w której dokonywane są obliczenia oraz przeszktałcenia pozwalające na obliczenie obrotu dłonie, odległości między wybranymi palcami oraz na wykrycie gestu. 
    
    % \inputminted[firstline=51, lastline=52]{python}{../OpenLeap.py}
    
    \lstinputlisting[language=python, firstline=51, lastline=52]{../OpenLeap.py}
    
    \quad W pierwszym kroku tworzymy instancję klasy \textbf{VideoCapture} biblioteki \textbf{OpenCV}, która pozwoli na odczytywanie obrazu kamery. 
    
    \lstinputlisting[language=python, firstline=272, lastline=297]{../OpenLeap.py}
    
    % \inputminted[firstline=272, lastline=297]{python}{../OpenLeap.py}
    
    %%%%%%%%%%%%%%%%%%%%%%%%%%%%%%%%%%%%%%%%%%%%%%%%%%%%%%%%%%%%%%%%%%%%
    %%%%%%%%%%%%%%%%%%%%%%%%%%%% MEDIA PIPE %%%%%%%%%%%%%%%%%%%%%%%%%%%%
    %%%%%%%%%%%%%%%%%%%%%%%%%%%%%%%%%%%%%%%%%%%%%%%%%%%%%%%%%%%%%%%%%%%%
    
    \subsection{MediaPipe - Elementy charakterystyczne}
    
    \quad Biblioteka MediaPipe o otwartym źródle, udostępnia wieloplatformowe oraz konfigurowalne rozwiązania wykrzustujące uczenie maszynowe w dziedzienie rozpoznawania, segmentacji oraz klasyfikacji obiektów wizji komputerowej. Niektórymi z rozwiązań są:
    
    \begin{itemize}
        \item Rozpoznawanie twarzy
        \item Segmentacjia włosów oraz twarzy
        \item Rozpoznawnie oraz określanie rozmiarów obiektów trójwymiarowych 
              na podstawie obrazu dwuwymiarowego. 
    \end{itemize}
    
    \quad Model modułu MediaPipe pozwala na wyznaczenie pozycji 21 elementów charakterystycznych dłoni. Współrzędne X i Y są znomralizowane względem rozdzielczości obrazu kamery. Współrzędna X względem liczby pikseli w osi X, a współrzędna Y względem liczby pikseli w osi Y. Oś Z jest prostopadła do osi X i Y, z punktem początkowym w punkcie określającym pozycję nadgarstka. Współrzędna Z jest znormalizowana względem szerokości obrazu kamery, tak jak współrzędna X. 
    
    \begin{figure}[H]
    \begin{center}
        \includegraphics[width=15cm]{../images/hand_landmarks.png}
        \caption{Elementy charakterystyczne dłoni}
    \end{center}
    \end{figure}
    
    \quad Pozycje nadgarstka, paliczków oraz stawów dłoni zostaną wykorzystane do obliczenia obrotu dłoni względem punkut 0 oraz do wytrenowania modeli uczenia maszynowego, których zadaniem będzie rozpoznawnie wybranych gestów. 
    
    \subsection{Generowanie grafiki dłoni}
    
    \quad Generowanie grafiki nałożonej na daną dłoń wykonuje się przy pomocy przygotowanej funkcji biblioteki MediaPipe, która współpracuje z OpenCV.   
    
    %%%%%%%%%%%%%%%%%%%%%%%%%%%%%%%%%%%%%%%%%%%%%%%%%%%%%%%%%%%%%%%%%%%%
    %%%%%%%%%%%%%%%%%%%%%%%%%%% SciKit Learn %%%%%%%%%%%%%%%%%%%%%%%%%%%
    %%%%%%%%%%%%%%%%%%%%%%%%%%%%%%%%%%%%%%%%%%%%%%%%%%%%%%%%%%%%%%%%%%%%
    
    
    \section{SciKit Learn - uczenie maszynowe}
    \subsection{Algorytm uczenia maszynowego - schemat UML}
    \quad SciKit Learn to biblioteka, która oferuje różnego typu metody uczenia maszynowego. Biblioteka zawiera algorytmy klasyfikacji, regresjii oraz analizy skupień. Przykładowym algorytmami w bibliotece są:
    
    \begin{itemize}
        \item Las losowy - polegająca na konstruowaniu wielu drzew decyzyjnych w czasie uczenia. 
        \item Algorytm centroidów - algorytm wykorzystywany w analizie skupień.
        \item Maszyna wektorów nośnoych - algorytm klasyfikujący, często wykorzystywany w procesie rozpoznawania obrazów. 
    \end{itemize}
    
    O wszystkich dostępnych algorytmach informacje można znaleźć w ogólnodostępnej dokumentacji biblioteki. 
    
    
    
    \subsection{Budowa programu}
    Celem programu będzie stworzenie modeli matematycznych przy pomocy metod uczenia maszynowego, których celem będzie rozpoznawanie gestów dłoni. Proces tworzenie takiego modelu można podzielić na trzy kroki.
    
    \begin{itemize}
        \item Zebranie i przetworzenie danych. 
        \item Przygotowanie algorytmów klasyfikacji i znalezenie najdokładniejszego. 
        \item Zapis modelu do pliku typu \textbf{pickle}
    \end{itemize}
    
    \subsection{Zebranie danych}
    
    \quad Przygotowanie danych do przetworzenia będzie wymagało paru opercaji matematycznych. Współrzędne opisujące pozycje elementów charakterystycznych są znormalizowane względem wielkości obrazu pobranego z kamery, z czego wynika, że środek układu współrzęnych jest w prawym górnym rogu obrazu pobranego z kamery. W takim wypadku należy przprowadzić transformację, tutaj akurat przesunięcie układu współrzędnych do pozycji nadgarstka. Taka operacja pozwoli na pozbycie się uniezależnienie zmiennych od pozycji elementów dłoni na obrazie. Ostatecznie pozbywamy się pozycji nadarstka z wektora danych, ponieważ jest ona środkiem nowego układu współrzędnych. 
    
    \quad \textbf{Macierz przesunięcia}
    
    \begin{equation*}
        M_p = 
        \begin{bmatrix}
        1 & 0 & 0 & -x_0 \\
        0 & 1 & 0 & -y_0 \\
        0 & 0 & 1 & -z_0 \\
        0 & 0 & 0 & 1
        \end{bmatrix}
    \end{equation*}
    
    \quad Aby dane mogły zostać zinterpretowane przez algorytmy ucznenia maszynowego muszą one zostać przedstawione w postaci jednowymiarowej. Aktualna postać macierzy przedstawiającej wpółrzędne elementów charakterystycznych ma następującą postać. Indeksy współrzędnych są równoznaczne z indeksami elementów dłoni. 
    
    \begin{equation*}
        M_p = 
        \begin{bmatrix}
        x_1' & y_1' & z_1' \\
        x_2' & y_2' & z_2' \\
         & \vdots &     \\
        x_{21}' & y_{21}' & z_{21}'
        \end{bmatrix}
    \end{equation*}
    
    Dane w postaci jednowymiarowej mają postać następującego wektora. 
    
    \begin{equation*}
        A_f=
        \begin{bmatrix}
            x_1 & y_1 & z_1 & x_2 & y_2 & \cdots & y_{21} & z_{21}
        \end{bmatrix}
    \end{equation*}
    
    
    \quad Drugim krokiem jest uniezależnienie pozycji elementów od odległości dłoni od kamery. Najprostszym rozwiązaniem jest normalizacja wektora danych względem największej bezwzględnej wartości. 
    
    \quad \textbf{Normalizacja}
    
    % \begin{equation*}
    %     A_n=
        % \begin{bmatrix}
        %     x_1 & y_1 & z_1 & x_2 & y_2 & \cdots & y_{21} & z_{21}
        % \end{bmatrix}
    % \end{equation*}
    
    
    \begin{equation*}
        A_n=\dfrac{A_f}{max(abs(A_f))}
    \end{equation*}
    
    \quad Każdy nowy wektor zostaje zapisany do pliku CSV z odpwowiednią etykietą. Zebrane dane posłużą do wytrenowania algorytmów uczenia maszynowego. 
    
    
    
    \subsection{Metody klasyfikacji - uczenie maszynowe}
    
    \quad Przygotowane dane zostają odczytane z pliku CSV. W pierwszym kroku należy rozdzielić je na dwie części: współrzędne (dane wejściowe) oraz etykiety (dane wyjściowe). W kolejnym kroku należy te dwie grupy podzielić na grupę trenującą i grupę testową. Zadaniem grupy testowej będzie trenowanie wybranych modeli matematycznych, a grupy testowej przetestowanie ich dokładności. 
    
    \quad W celu wybrania najlepszej metody klasyfikacji, zostnie wybranych kilka algorytmów. Każdy z nich stworzy swój model, a ostatecznie zostanie sprawdzona ich poprawności z wykorzystaniem grupy testowej. Model z najepszym wynikiem zostanie zapisany do pliku typu \textbf{pickle}. W języku Python pliki typu \textbf{pickle} pozwalają na zapis zmiennych, obiektów lub innych struktur danych, które mają zostać wykorzystane w po zakończeniu programu. 
    
    \subsection{Ponowne wykorzystanie modelu}
    
    \quad Gotowy model pobieramy i testujemy w przykładowym programie. 
    
    \section{Paczka PyPi}
    \subsection{Budowa paczki}
    
    \quad Ostatecznym krokiem jest przygotowanie programu w formie paczki, która zostanie udostępniona na platformie PyPi. Wygama to przygotowania odpowiednich plików konfiguracyjnych oraz zostosowania stosownych narzędzi do stworznie pliku \textbf{wheel} oraz \textbf{tar}. 
    
    \subsection{Struktura Paczki}
    \quad Pierwszm krokiem jest przygotowanie odpowiedniej struktury paczki. Do tego celu został stworzony folder o poniższej strukturze. W tym folderze znajdują się wszystkie potrzebne elementy paczki. W podfolderze o tej samej nazwie znajduje się główna części modułu, czyli plik .py, w którym zapisana jest klasa OpenLeap. Dodatkowo w tym folderze znajdują się pliki typu \textbf{pickle}, w których zapisane są modele rozpoznające gesty.
    
    \begin{figure}
    \centering
        \begin{minipage}{7cm}
            \dirtree{%
            .1 openleap.
            .2 openleap.
            .3 \hyperref[openleap-file1]{\_\_init\_\_.py}.
            .3 \hyperref[openleap-file2]{OpenLeap.py}.
            .3 \hyperref[openleap-file3]{gesture\_recognition.pkl}.
            .3 \hyperref[openleap-file4]{sign\_language\_alphabet.pkl}.
            .2 LICENSE.
            .2 MANIFEST.
            .2 README.md.
            .2 setup.py.
            } 
        \end{minipage}
        \caption{Struktura paczki PyPi}
    \end{figure}
    
    \subsection{Pliki Konfiguracyjne}
    \quad Pliki setup.py oraz MANIFEST są plikami, które odpowiadają za konfigurację oraz opis paczki. W pliku setup.py zapisany jest numer aktualnej wersji, autor, kontakt do autora, nazwa paczki itp. 
    
    \quad 
    
    
    % \subsection{Plik setup.py}
    \subsection{Załadowanie paczki do repozytorium}
    
    \quad Przed załadowaniem paczki do repozytorium, należy stworzyć zapakowaną paczkę źródłową, na przykład typu .tar oraz plik typu WHEEL. Oba pliki spełniają tą samą funkcję, czyli przechowywnie niezbędnych elementów paczki oraz umożliwiają ich instalację na systemie użytkownika. Plik WHEEL pozwala na dużo szybszy proces instalacji niż instalacja ze źródła, czyli paczki typu .tar. 


%%%%%%%%%%%%%%% Weryfikacja i Walidacje %%%%%%%%%%%%%%%%
\chapter{Weryfikacja i walidacja}
\begin{itemize}
\item sposób testowania w ramach pracy (np. odniesienie do modelu V)
\item organizacja eksperymentów
\item przypadki testowe zakres testowania (pełny/niepełny)
\item wykryte i usunięte błędy
\item opcjonalnie wyniki badań eksperymentalnych
\end{itemize}

\begin{table}
\centering
\caption{Opis tabeli nad nią.}
\label{id:tab:wyniki}
\begin{tabular}{rrrrrrrr}
\toprule
	         &                                     \multicolumn{7}{c}{metoda}                                      \\
	         \cmidrule{2-8}
	         &         &         &        \multicolumn{3}{c}{alg. 3}        & \multicolumn{2}{c}{alg. 4, $\gamma = 2$} \\
	         \cmidrule(r){4-6}\cmidrule(r){7-8}
	$\zeta$ &     alg. 1 &   alg. 2 & $\alpha= 1.5$ & $\alpha= 2$ & $\alpha= 3$ &   $\beta = 0.1$  &   $\beta = -0.1$ \\
\midrule
	       0 &  8.3250 & 1.45305 &       7.5791 &    14.8517 &    20.0028 & 1.16396 &                       1.1365 \\
	       5 &  0.6111 & 2.27126 &       6.9952 &    13.8560 &    18.6064 & 1.18659 &                       1.1630 \\
	      10 & 11.6126 & 2.69218 &       6.2520 &    12.5202 &    16.8278 & 1.23180 &                       1.2045 \\
	      15 &  0.5665 & 2.95046 &       5.7753 &    11.4588 &    15.4837 & 1.25131 &                       1.2614 \\
	      20 & 15.8728 & 3.07225 &       5.3071 &    10.3935 &    13.8738 & 1.25307 &                       1.2217 \\
	      25 &  0.9791 & 3.19034 &       5.4575 &     9.9533 &    13.0721 & 1.27104 &                       1.2640 \\
	      30 &  2.0228 & 3.27474 &       5.7461 &     9.7164 &    12.2637 & 1.33404 &                       1.3209 \\
	      35 & 13.4210 & 3.36086 &       6.6735 &    10.0442 &    12.0270 & 1.35385 &                       1.3059 \\
	      40 & 13.2226 & 3.36420 &       7.7248 &    10.4495 &    12.0379 & 1.34919 &                       1.2768 \\
	      45 & 12.8445 & 3.47436 &       8.5539 &    10.8552 &    12.2773 & 1.42303 &                       1.4362 \\
	      50 & 12.9245 & 3.58228 &       9.2702 &    11.2183 &    12.3990 & 1.40922 &                       1.3724 \\
\bottomrule
\end{tabular}
\end{table}  
 

%%%%%%%%%%%%%%%%%% Podsumowanie %%%%%%%%%%%%%%%%%%%
\chapter{Podsumowanie i wnioski}
\begin{itemize}
\item uzyskane wyniki w świetle postawionych celów i zdefiniowanych wyżej wymagań
\item kierunki ewentualnych danych prac (rozbudowa funkcjonalna …)
\item problemy napotkane w trakcie pracy
\end{itemize}

 
\bibliographystyle{plplain}
\bibliography{bibliografia}
\addcontentsline{toc}{chapter}{Bibliografia}


\begin{appendices}
 

\chapter*{Spis skrótów i symboli}
\addcontentsline{toc}{chapter}{Spis skrótów i symboli}

\begin{itemize}
\item[DNA] kwas deoksyrybonukleinowy (ang. \ang{deoxyribonucleic acid})
\item[MVC] model -- widok -- kontroler (ang. \ang{model--view--controller}) 
\item[$N$] liczebność zbioru danych
\item[$\mu$] stopnień przyleżności do zbioru
\item[$\mathbb{E}$] zbiór krawędzi grafu
\item[$\mathcal{L}$] transformata Laplace'a 
\end{itemize}


\chapter*{Źródła}
\addcontentsline{toc}{chapter}{Źródła}

Jeżeli w pracy konieczne jest umieszczenie długich fragmentów kodu źródłowego, należy je przenieść do załącznika.


\chapter*{Zawartość dołączonej płyty}
\addcontentsline{toc}{chapter}{Zawartość dołączonej płyty}

Do pracy dołączona jest płyta CD z~następującą zawartością:
\begin{itemize}
\item praca (źródła \LaTeX owe i końcowa wersja w \texttt{pdf}),
\item źródła programu,
\item dane testowe.
\end{itemize}

\listoffigures
\listoftables
	
\end{appendices}


\end{document}


%% Finis coronat opus.
