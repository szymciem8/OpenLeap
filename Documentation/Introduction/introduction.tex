\chapter{Wstęp}
\begin{itemize}
\item wprowadzenie w problem/zagadnienie
\item osadzenie problemu w dziedzinie
\item cel pracy
\item zakres pracy
\item zwięzła charakterystyka rozdziałów
\item jednoznaczne określenie wkładu autora, w przypadku prac wieloosobowych – tabela z autorstwem poszczególnych elementów pracy
\end{itemize}

\newpage

\section{Wprowawdzenie w problem}
\quad Rozwój technolgii w ostatnich czasach przyczynił się do coraz częstszego wykorzystywania wizji komputerowej oraz metod uczenia maszynowego do rozpoznawania oraz klasyfikacji różnego typu obiektów, w tym części ludzkiego ciała. Pozwala to na interakcję człowieka z aplikacjami, często w spób bardziej naturalny. 

\section{Cel pracy}
Projekt inżynierski ma na celu stworzenie biblioteki w języku Python, która pozwoli na przystępne wykorzystanie algorytmów rozpoznawania gestów oraz ruchu dłoni. Biblioteka powinna oferować gotowe rozwiązania, na przykład przygotowane modele matematyczne pozwalające na rozpoznawanie języka migowego oraz gestów podstawowych. Dodatkowo powinna pozowlić na wyznaczenie pozycji dłoni, jej typu oraz wartości charakterystycznych, na przykład odlgłości między końcówkami wybranych palców czy kąta obrotu dłoni. 

\section{Osadzenie problemu w dziedzienie}
\quad Aktualnie istnieją częściowo gotowe rozwiązania pozwalające na rozpoznanie i klasyfikację dłoni - bibliotka MediaPipe. 

\section {Charakterystyka rozdziałów}

% \section{Ogólnie}

% Praca przedstawia wykorzystanie bibliotek OpenCV, MediaPipe oraz metod 
% uczenia maszynowego bibliteki SciKit Learn do stworzenia modułu dla 
% języka Python, który umożliwi sterwowanie dowolnymi aplikacjami przy 
% użyciu gestów oraz ruchu dłońmi. 

% \section{Wykorzystane technologie}
% \subsection{OpenCV}
% Biblioteka, dzieki której można wykorzystać obraz z kamery oraz wstępnie
% przetworzyć obraz, który zostanie wykorzystany przez bibliotekę MediaPipe.

% \subsection{MediaPipe}

% \quad
% Biblioteka MediaPipe o otwartym źródle, udostępnia wieloplatformowe oraz 
% konfigurowalne rozwiązania wykrzustujące uczenie maszynowe w dziedzienie 
% rozpoznawania, segmentacji oraz klasyfikacji obiektów wizji komputerowej. 
% Niektórymi z rozwiązań są:

% \begin{itemize}
%     \item Rozpoznawanie twarzy
%     \item Segmentacjia włosów oraz twarzy
%     \item Rozpoznawnie oraz określanie rozmiarów obiektów trójwymiarowych na podstawie obrazu dwuwymiarowego. 
% \end{itemize}

% Platformy/Języki programowania obsługiwane przez MediaPipe:

% \begin{itemize}
%     \item Android
%     \item IOS
%     \item JavaScript
%     \item Python
%     \item C++
%     \item Coral
% \end{itemize}

% ????
% Pozwoli na rozpoznaie dłoni oraz jej elementów charakterystycznych, 
% takich jak nagdarstek, stawy oraz końcówki palców. 
% ???


% \subsection{SciKit Learn}

% SciKit Learn to biblioteka, która oferuje różnego typu metody uczenia
% maszynowego. Biblioteka zawiera algorytmy klasyfikacji, regresjii oraz 
% analizy skupień. 
% Biblioteka pozwala na wykorzystanie metod klasyfikacji uczenia maszynowego. 
% Co pozwli na rozpoznanie gestów dłoni. 
