\chapter{Wstęp}

\section{Wprowadzenie w problem}
\quad Rozwój technologii w ostatnich czasach przyczynił się do coraz częstszego wykorzystywania wizji komputerowej i metod uczenia maszynowego do rozpoznawania oraz klasyfikacji różnego typu obiektów, w tym części ludzkiego ciała. Pozwala to na interakcję człowieka z aplikacjami, często w sposób bardziej naturalny i intuicyjny. 

\section{Cel pracy}
Projekt inżynierski ma na celu stworzenie biblioteki w języku Python, która pozwoli na przystępne wykorzystanie algorytmów rozpoznawania gestów oraz ruchu dłoni. Biblioteka powinna oferować gotowe rozwiązania, na przykład przygotowane modele matematyczne pozwalające na rozpoznawanie alfabetu języka migowego oraz podstawowych gestów (otwarta, zamknięta dłoń). Dodatkowo powinna pozwolić na wyznaczenie pozycji dłoni, jej typu oraz innych własności takich jak, odległości między końcówkami wybranych palców czy kąta obrotu dłoni. 

\newpage
\section {Charakterystyka rozdziałów}

\begin{itemize}
    \item \textbf{Analiza tematu} -- Opis założeń technicznych oraz funkcyjnych całości systemu. 
    \item \textbf{Wymagania i narzędzia} -- Wybrane narzędzia, w tym system operacyjny, biblioteki oraz programy. 
    \item \textbf{Specyfikacja zewnętrzna} -- Działanie oraz obsługa programu z perspektywy użytkownika. 
    \item \textbf{Specyfikacja wewnętrzna} -- Opis budowy całości paczki, wykorzystanych metod, struktur danych oraz algorytmów.
    \item \textbf{Podsumowanie} -- Podsumowanie całości projektu. \newline
\end{itemize}
