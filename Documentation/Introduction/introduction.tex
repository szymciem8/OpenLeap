\chapter{Wstęp}
% \begin{itemize}
% \item wprowadzenie w problem/zagadnienie
% \item osadzenie problemu w dziedzinie
% \item cel pracy
% \item zakres pracy
% \item zwięzła charakterystyka rozdziałów
% \item jednoznaczne określenie wkładu autora, w przypadku prac wieloosobowych – tabela z autorstwem poszczególnych elementów pracy
% \end{itemize}

\section{Wprowadzenie w problem}
\quad Rozwój technologii w ostatnich czasach przyczynił się do coraz częstszego wykorzystywania wizji komputerowej i metod uczenia maszynowego do rozpoznawania oraz klasyfikacji różnego typu obiektów, w tym części ludzkiego ciała. Pozwala to na interakcję człowieka z aplikacjami, często w sposób bardziej naturalny i intuicyjny. 

\section{Cel pracy}
Projekt inżynierski ma na celu stworzenie biblioteki w języku Python, która pozwoli na przystępne wykorzystanie algorytmów rozpoznawania gestów oraz ruchu dłoni. Biblioteka powinna oferować gotowe rozwiązania, na przykład przygotowane modele matematyczne pozwalające na rozpoznawanie alfabetu języka migowego oraz podstawowych gestów (otwarta, zamknięta dłoń). Dodatkowo powinna pozwolić na wyznaczenie pozycji dłoni, jej typu oraz innych własności takich jak, odległości między końcówkami wybranych palców czy kąta obrotu dłoni. 

% \section{Osadzenie problemu w dziedzienie}
% \quad Aktualnie istnieją częściowo gotowe rozwiązania pozwalające na rozpoznanie i klasyfikację dłoni -- bibliotka MediaPipe. 

\newpage
\section {Charakterystyka rozdziałów}

\begin{itemize}
    \item \textbf{Analiza tematu} -- Opis założeń technicznych oraz funkcyjnych całości systemu. 
    \item \textbf{Wymagania i narzędzia} -- Wybrane narzędzia, w tym system operacyjny, biblioteki oraz programy. 
    \item \textbf{Specyfikacja zewnętrzna} -- Działanie oraz obsługa programu z perspektywy użytkownika. 
    \item \textbf{Specyfikacja wewnętrzna} -- Opis budowy całości paczki, wykorzystanych metod, struktur danych oraz algorytmów.
    \item \textbf{Podsumowanie} -- Podsumowanie całości projektu. \newline
\end{itemize}

% \quad Rozdział \enquote{Analiza tematu} będzie opisywał sformułowanie t

% \quad W pierwszym rozdziale zostanie poruszony temat genezy problemu oraz jego sformułowania. Zostaną dodatkowo opisane jego założenia wraz z wyamaganą funkcjonalnością projektu. 

% \quad W drugim temacie zostaną opisne techniczne tworzonej biblioteki oraz narzędzia wymagane do jej stworzenia. Zakres opisywanych narzędzi rozpocznie się od opisu wybranego systemu operacyjnego do programów służących do stworzenia pobieralnej paczki na platformi PyPi.

% \quad W rozdziale nr 5 zostaną opisane wymagania sprzętowe wymagane do poprawnego działania funkcji biblioteki. Przedstawione zostanie działanie klasy wraz z jej parametrami oraz sposób jego wykorzysania. W drugiej jego części zostaną przedstawione przykłady wykorzysania modułu. Przykładowym projektami będą: program rozpoznający gest, interaktywny kiosk bezdotykowy i system doboru koloru przy pomocy gestu. 

% \quad Kolejny rozdział opisuje budowę klasy wraz z najważniejszymi algorytmami oraz strukturami danych. Dodakowo opisany zostanie Jupyter Notebook służący do generowania modeli uczenia maszynowego rozpoznających gest dłoni. 

% \section{Ogólnie}

% Praca przedstawia wykorzystanie bibliotek OpenCV, MediaPipe oraz metod 
% uczenia maszynowego bibliteki SciKit Learn do stworzenia modułu dla 
% języka Python, który umożliwi sterwowanie dowolnymi aplikacjami przy 
% użyciu gestów oraz ruchu dłońmi. 

% \section{Wykorzystane technologie}
% \subsection{OpenCV}
% Biblioteka, dzieki której można wykorzystać obraz z kamery oraz wstępnie
% przetworzyć obraz, który zostanie wykorzystany przez bibliotekę MediaPipe.

% \subsection{MediaPipe}

% \quad
% Biblioteka MediaPipe o otwartym źródle, udostępnia wieloplatformowe oraz 
% konfigurowalne rozwiązania wykrzustujące uczenie maszynowe w dziedzienie 
% rozpoznawania, segmentacji oraz klasyfikacji obiektów wizji komputerowej. 
% Niektórymi z rozwiązań są:

% \begin{itemize}
%     \item Rozpoznawanie twarzy
%     \item Segmentacjia włosów oraz twarzy
%     \item Rozpoznawnie oraz określanie rozmiarów obiektów trójwymiarowych na podstawie obrazu dwuwymiarowego. 
% \end{itemize}

% Platformy/Języki programowania obsługiwane przez MediaPipe:

% \begin{itemize}
%     \item Android
%     \item IOS
%     \item JavaScript
%     \item Python
%     \item C++
%     \item Coral
% \end{itemize}

% ????
% Pozwoli na rozpoznaie dłoni oraz jej elementów charakterystycznych, 
% takich jak nagdarstek, stawy oraz końcówki palców. 
% ???


% \subsection{SciKit Learn}

% SciKit Learn to biblioteka, która oferuje różnego typu metody uczenia
% maszynowego. Biblioteka zawiera algorytmy klasyfikacji, regresjii oraz 
% analizy skupień. 
% Biblioteka pozwala na wykorzystanie metod klasyfikacji uczenia maszynowego. 
% Co pozwli na rozpoznanie gestów dłoni. 
