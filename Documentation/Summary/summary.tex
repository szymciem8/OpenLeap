\chapter{Podsumowanie i wnioski}

% \begin{itemize}
% \item uzyskane wyniki w świetle postawionych celów i zdefiniowanych wyżej wymagań
% \item kierunki ewentualnych danych prac (rozbudowa funkcjonalna …)
% \item problemy napotkane w trakcie pracy
% \end{itemize}

\quad Podsumowując, celem pracy było stworzenie biblioteki, która pozwoli na rozszerzenie aplikacji pisanych w języku Python o możliwość sterowania dłońmi. Biblioteka jest o otwartym kodzie źródłowym, co oznacza, że można z niej bezpłatnie korzystać oraz każdy może dołączyć do jej rozbudowywania.  

\quad Bibliotekę można ulepszyć poprzez napisanie jej w języku C++, który jest kompilowany. Pozwoliłoby to na zwiększenie wydajności oprogramowania. Język Python zezwala na korzystanie z modułów napisanych właśnie w języku C i C++, więc jest to język jak najbardziej kompatybilny. 