\chapter{Wymagania i narzędzia}

\begin{itemize}
\item wymagania funkcjonalne i niefunkcjonalne
\item przypadki użycia (diagramy UML) - dla prac, w których mają zastosowanie
\item opis narzędzi, metod eksperymentalnych, metod modelowania itp.
\item metodyka pracy nad projektowaniem i implementacją - dla prac, w których ma to zastosowanie
\end{itemize}


\section{Wymagania}

\quad Klasa powinna zawierać w sobie wszystkie niezbędne funkcje oraz parametry pozwalające na wykorzysatnie jej w dowolnym projekcie. Modele rozpoznające gesty powinny zostać uprzednio przygotowane. 

\quad Pobranie modułu powinno być możliwe poprzez wykorzystanie standardowego programu do zarządzania paczkami \textbf{pip}. Ten menedżer pozwala na instalacje paczki wraz z jej zależnościami, czyli innymi paczkami wymaganami do porwanego działania pobieranego modułu. 

\section{Narzędzia}

\subsection{System operacyjny Ubuntu}
\quad Do stworzenia oprogramowania została wybrana dystrybucja Ubuntu systemu Linux. Dzięki takim elementom jak menedżer pakietów lub wbudowane wspracie dla języka Python.

\subsection{Python}
\quad Do napisanie biblioteki został wykorzystany język skryptowy Python. Ze względu na swoją budowę, Python nie należy do najbardziej wydajnych języków. Python jest językiem interpretowalnym, co oznacza, że jego program nie zostaje przekompilowany do kodu maszynowego, a zostaje on zinterpretowany przez program nazywany interpreterem. 
\quad Pomimo niskiej wydajności języka Python, jest on dobrym narzędziem do tworzenia aplikacji wykorzystujących wizję komputerową oraz uczenia maszynowe. Głównie ze względu na dostępność oraz jakość bibliotek o wolnym źródle.  

\subsection{iPython}
\quad To iteraktywna powłoka dla języka Python, która rozszerza jego działanie o interospekcję, czyli możliwość wykonywania poprzednich części programu zapisanych w komórkach. Dodatkowo IPython oferuje dodatkową składnię powołoki oraz wykonywanie komend wiersza poleceń. 

\subsection{Jupyter Notebook}
\quad Dodatkowym narzędziem jest Jupyter Notebook, który pozwala na wykonwyanie części programu w osobnych komórkach. Pozwala to na łatwiejsze odnajdowanie błędów w programie.
\quad Jupyter Notebook został wykorzystany do napisania programu trenującego modele uczenia mszynowego rozponzące gesty. Dzięki strukturze takiego pliku, można w prosty sposób wykonywać program w danej komórce wiele razy. Na przykład pobieranie danych, dla różnych gestów. 

\subsection{Visual Studio Code - Środowisko Pracy}
\quad Całość klasy została napisana z wykorzystaniem edytora Visual Studio Code. Ten edytor został wybrany ze względu na jego wielozadaniowość. Visual Studio Code obsługuje jednocześnie programy napisane w Pythonie oraz pliki z rozszerzeniem \textbf{.ipynb}, czyli tak zwane zeszyty, które wykorzystują interaktywnego Pythona, czyli język iPython. 

\subsection{Platforma PyPi}
\quad Platforma PyPi jest platformą, na której udostępniane są modułu napisane w języku Python. Aktualnie jest ona zarządzana poprzez fundację "Python Software Foundation". Dzięki tej platfomie, zupełnie za darmo można pobierać znajdujące się na niej oprogramowanie oraz udostępniać własne. Program \textbf{pip} w głónwej mierze korzysta z PyPi jako głównego repozytorium z paczkami. 

\subsection{Programy bdist\_wheel oraz sdist}
\quad Oba programy są niezbędne do przygotowania w pełni działającej paczki udostępnionej przez platforme PyPi. Program \textbf{sdist} pozwala na stworzenie źródłowej dystrubucji, czyli w praktce pliku typu \textbf{.zip}, \textbf{.tar} czy też \textbf{.rar} oraz wielu innych, w których znajdują się wybrane pliki. 
\quad Kolejnym programem jest \textbf{bdist\_wheel}, który odpowiada za stworzenie uprzednio paczki typu \textbf{WHEEL}, która w przypadku wykorzystania plików napisanych w języku \textbf{C} uprzednio je kompiluje, dzięki czemu nie jest wymagany kompilator po stronie użytkownika. Ostatecznie, \textbf{WHEEL} pozwala na szybszą instalację paczki w porównaniu z budowaniem jej z plików źródłowych. 


% %%%%%%%%%%%%%%%%%%%%%%%%%%%%%%%%%%%%%%%%%%%%%%%%%%%%%%%%%%%%%%%%%%%%
% %%%%%%%%%%%%%%%%%%%%%%%%%%%%%% OPEN CV %%%%%%%%%%%%%%%%%%%%%%%%%%%%%
% %%%%%%%%%%%%%%%%%%%%%%%%%%%%%%%%%%%%%%%%%%%%%%%%%%%%%%%%%%%%%%%%%%%%
% \subsection{OpenCV - przygotowanie obrazu z kamery}


% %%%%%%%%%%%%%%%%%%%%%%%%%%%%%%%%%%%%%%%%%%%%%%%%%%%%%%%%%%%%%%%%%%%%
% %%%%%%%%%%%%%%%%%%%%%%%%%%%% MEDIA PIPE %%%%%%%%%%%%%%%%%%%%%%%%%%%%
% %%%%%%%%%%%%%%%%%%%%%%%%%%%%%%%%%%%%%%%%%%%%%%%%%%%%%%%%%%%%%%%%%%%%

% \subsection{MediaPipe - Elementy charakterystyczne}

% \subsection{Generowanie grafiki dłoni}

% %%%%%%%%%%%%%%%%%%%%%%%%%%%%%%%%%%%%%%%%%%%%%%%%%%%%%%%%%%%%%%%%%%%%
% %%%%%%%%%%%%%%%%%%%%%%%%%%% SciKit Learn %%%%%%%%%%%%%%%%%%%%%%%%%%%
% %%%%%%%%%%%%%%%%%%%%%%%%%%%%%%%%%%%%%%%%%%%%%%%%%%%%%%%%%%%%%%%%%%%%

% \section{SciKit Learn - uczenie maszynowe}

