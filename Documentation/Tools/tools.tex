\chapter{Wymagania i narzędzia}

% \begin{itemize}
% \item wymagania funkcjonalne i niefunkcjonalne
% \item przypadki użycia (diagramy UML) - dla prac, w których mają zastosowanie
% \item opis narzędzi, metod eksperymentalnych, metod modelowania itp.
% \item metodyka pracy nad projektowaniem i implementacją - dla prac, w których ma to zastosowanie
% \end{itemize}


\section{Wymagania Funkcjonalne i Niefunkcjonalne}

\quad Klasa powinna zawierać w sobie wszystkie niezbędne funkcje oraz parametry pozwalające na wykorzystanie jej w dowolnym projekcie, takie jak na przykład obliczanie kąta obrotu dłoni względem nadgarstka czy rozpoznawanie gestów. Dwa podstawowe modele rozpoznające gesty powinny zostać uprzednio przygotowane, natomiast użytkownik powinien mieć dodatkowo możliwość stworzenia własnego modelu rozpoznającego wybrane ułożenie dłoni. 

\quad Pobranie modułu powinno być możliwe poprzez wykorzystanie standardowego menedżera do zarządzania paczkami w języku Python, czyli programu \textbf{pip}. Ten menedżer pozwala na instalacje paczki wraz z jej zależnościami, czyli innymi paczkami wymaganiami do poprawnego działania pobieranej biblioteki. 

\quad Na głównej stronie projektu powinna zostać zamieszczona krótka dokumentacja w postaci pliku \textbf{README.md}, w którym zostanie opisany proces instalacji paczki, jej funkcjonalność oraz przykładowe programy. 

\section{Narzędzia}

\quad Przy tworzeniu modułu ważne są wykorzystywane narzędzia, zaczynając od wybranego systemu operacyjnego, a kończąc na programach pozwalających na przygotowanie plików źródłowych wysyłanych na zdalne repozytorium PyPi, każde z nich jest kluczowe do zbudowania pełnego projektu. 

\subsection{System operacyjny Ubuntu}
\quad Do stworzenia oprogramowania została wybrana dystrybucja Ubuntu w wersji 20.04 LTS. System został wybrany ze względu na istnienie takich elementów jak powłoka systemowa \textbf{bash}, menedżer pakietów oraz wsparcie dla języka Python. 

\subsection{GitHub}
\quad Platforma GitHub, która wykorzystuje rozproszony system kontroli Git, pozwala na stworzenie głównej strony biblioteki, na której znajdą się pliki programu wraz z dokumentacją oraz instrukcją instalacji i korzystania. W czasie pracy nad modułem system Git pomaga w tworzeniu kopii zapasowych. W przypadku dalszego rozwoju projektu GitHub może posłużyć jako medium pracy innych kontrybutorów. 

\subsection{Python}
\quad Do napisania biblioteki został wykorzystany język skryptowy Python. Dzięki swojej popularności oraz dostępności, Python stał się językiem powszechnie stosowanym w pracy związanej z zagadnieniami uczenia maszynowego, analizy danych oraz przetwarzania obrazów. Na platformie PyPi można znaleźć wiele narzędzi pozwalających na pracę właśnie w tych dziedzinach, jak i również wielu innych. 

\quad Ze względu na swoją budowę, Python nie należy do najwydajniejszych języków. Python jest językiem interpretowanym, co oznacza, że jego program nie zostaje skompilowany do kodu maszynowego, a zostaje on zinterpretowany przez interpreter. Ma ta jednak swoje zalety, skrypt można wykonywać w częściach, co pozwala na testowanie działania programu. Dzięki czemu łatwiej jest znajdować błędy i na bieżąco je likwidować. Potencjał tego w pełni wykorzystuje interaktywna powłoka iPython. 

\subsection{iPython}
\quad To interaktywna powłoka dla języka Python, która rozszerza jego działanie o introspekcję, czyli możliwość wykonywania poprzednich części programu. W praktyce oznacza to, że użytkownik ma możliwość wykonywania programu zawartego w komórkach w dowolnej kolejności, nawet jeśli oznacza to wykonywanie programu zapisanego w komórkach poprzedzających aktualną. 

\quad Dodatkowo iPython oferuje możliwość korzystania z komend wiersza poleceń. Pozwala to, na przykład na instalowanie modułów wewnątrz programu. 

\subsection{Jupyter Notebook}

\quad Do obsługi interaktywnego języka Python, wykorzystany został Jupyter Notebook (\enquote{notatnik}), czyli webowy edytor, który został stworzony z myślą o pracy związanej z analizą danych oraz obliczeniami naukowymi. 

\quad Notatnik pozwala na zapis tekstu w języku Markdown, czyli języku znaczników stosowanym do formatowania tekstu. Co pozwala na czytelny opis programu oraz wstawianie obrazów, jeśli są wymagane. Dodatkowo wszystkie wykresy generowane przez na przykład bibliotekę \textbf{matplotlib} będą wyświetlane na stałe pod komórką, w której został wykonany program. 

\subsection{Visual Studio Code - Środowisko Pracy}
\quad Ten edytor został wybrany ze względu na jego wielozadaniowość. Visual Studio Code obsługuje jednocześnie programy napisane w Pythonie oraz pliki z rozszerzeniem \textbf{.ipynb}, czyli tak zwane zeszyty, które wykorzystują interaktywnego Pythona, czyli język iPython. 

\subsection{Platforma PyPi}
\quad Platforma PyPi jest platformą, na której udostępniane są modułu napisane w języku Python. Aktualnie jest ona zarządzana poprzez fundację "Python Software Foundation". Dzięki tej platfomie, zupełnie za darmo można pobierać znajdujące się na niej oprogramowanie oraz udostępniać własne. Program \textbf{pip} w głónwej mierze korzysta z PyPi jako głównego repozytorium z paczkami. 

\subsection{Programy bdist\_wheel oraz sdist}
\quad Oba programy są niezbędne do przygotowania w pełni działającej paczki udostępnionej przez platforme PyPi. Program \textbf{sdist} pozwala na stworzenie źródłowej dystrubucji, czyli w praktce pliku typu \textbf{.zip}, \textbf{.tar} czy też \textbf{.rar} oraz wielu innych, w których znajdują się wybrane pliki. 

\quad Kolejnym programem jest \textbf{bdist\_wheel}, który odpowiada za stworzenie uprzednio paczki typu \textbf{WHEEL}, która w przypadku wykorzystania plików napisanych w języku \textbf{C} uprzednio je kompiluje, dzięki czemu nie jest wymagany kompilator po stronie użytkownika. Ostatecznie, \textbf{WHEEL} pozwala na szybszą instalację paczki w porównaniu z budowaniem jej z plików źródłowych. 


% %%%%%%%%%%%%%%%%%%%%%%%%%%%%%%%%%%%%%%%%%%%%%%%%%%%%%%%%%%%%%%%%%%%%
% %%%%%%%%%%%%%%%%%%%%%%%%%%%%%% OPEN CV %%%%%%%%%%%%%%%%%%%%%%%%%%%%%
% %%%%%%%%%%%%%%%%%%%%%%%%%%%%%%%%%%%%%%%%%%%%%%%%%%%%%%%%%%%%%%%%%%%%
% \subsection{OpenCV - przygotowanie obrazu z kamery}


% %%%%%%%%%%%%%%%%%%%%%%%%%%%%%%%%%%%%%%%%%%%%%%%%%%%%%%%%%%%%%%%%%%%%
% %%%%%%%%%%%%%%%%%%%%%%%%%%%% MEDIA PIPE %%%%%%%%%%%%%%%%%%%%%%%%%%%%
% %%%%%%%%%%%%%%%%%%%%%%%%%%%%%%%%%%%%%%%%%%%%%%%%%%%%%%%%%%%%%%%%%%%%

% \subsection{MediaPipe - Elementy charakterystyczne}

% \subsection{Generowanie grafiki dłoni}

% %%%%%%%%%%%%%%%%%%%%%%%%%%%%%%%%%%%%%%%%%%%%%%%%%%%%%%%%%%%%%%%%%%%%
% %%%%%%%%%%%%%%%%%%%%%%%%%%% SciKit Learn %%%%%%%%%%%%%%%%%%%%%%%%%%%
% %%%%%%%%%%%%%%%%%%%%%%%%%%%%%%%%%%%%%%%%%%%%%%%%%%%%%%%%%%%%%%%%%%%%

% \section{SciKit Learn - uczenie maszynowe}

\section{Biblioteki}

\subsection{OpenCV}

\subsection{MediaPipe}

\quad Biblioteka MediaPipe o otwartym źródle, udostępnia wieloplatformowe oraz konfigurowalne rozwiązania wykrzustujące uczenie maszynowe w dziedzienie rozpoznawania, segmentacji oraz klasyfikacji obiektów wizji komputerowej. Niektórymi z rozwiązań są:

\begin{itemize}
    \item Rozpoznawanie twarzy
    \item Segmentacjia włosów oraz twarzy
    \item Rozpoznawnie oraz określanie rozmiarów obiektów trójwymiarowych 
            na podstawie obrazu dwuwymiarowego. 
\end{itemize}

\quad Model modułu MediaPipe pozwala na wyznaczenie pozycji 21 elementów charakterystycznych dłoni. Współrzędne X i Y są znomralizowane względem rozdzielczości obrazu kamery. Współrzędna X względem liczby pikseli w osi X, a współrzędna Y względem liczby pikseli w osi Y. Oś Z jest prostopadła do osi X i Y, z punktem początkowym w punkcie określającym pozycję nadgarstka. Współrzędna Z jest znormalizowana względem szerokości obrazu kamery, tak jak współrzędna X. 

\subsection{SciKit Learn}

\quad SciKit Learn to biblioteka, która oferuje różnego typu metody uczenia maszynowego. Biblioteka zawiera algorytmy klasyfikacji, regresjii oraz analizy skupień. Przykładowym algorytmami w bibliotece są:
\begin{itemize}
    \item Las losowy - polegająca na konstruowaniu wielu drzew decyzyjnych w czasie uczenia. 
    \item Algorytm centroidów - algorytm wykorzystywany w analizie skupień.
    \item Maszyna wektorów nośnoych - algorytm klasyfikujący, często wykorzystywany w procesie rozpoznawania obrazów. 
\end{itemize}
O wszystkich dostępnych algorytmach informacje można znaleźć w ogólnodostępnej dokumentacji biblioteki. 


