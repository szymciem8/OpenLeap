\chapter{Wstęp}

\section{Ogólnie}

Praca przedstawia wykorzystanie bibliotek OpenCV, MediaPipe oraz metod 
uczenia maszynowego biblitoeki SciKit Learn do stworzenia modułu dla 
języka Python, który umożliwi sterwowanie dowolnymi aplikacjami przy 
użyciu gestów oraz ruchu dłoni.

\section{Wykorzystane technologie}
\subsection{OpenCV}
Biblioteka, dzieki której można wykorzystać obraz z kamery oraz wstępnie
przetworzyć obraz, który zostanie wykorzystany przez bibliotekę MediaPipe.

\subsection{MediaPipe}
Pozwoli na rozpoznaie dłoni oraz jej elementów charakterystycznych, 
takich jak nagdarstek, stawy oraz końcówki palców. 

\subsection{SciKit Learn}
Biblioteka pozwala na wykorzystanie metod klasyfikacji uczenia maszynowego. 
Co pozwli na rozpoznanie gestów dłoni. 
