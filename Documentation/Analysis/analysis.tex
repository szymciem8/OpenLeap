\chapter{Analiza Tematu}

% \begin{itemize}
% \item sformułowanie problemu
% \item osadzenie tematu w kontekście aktualnego stanu wiedzy ( state of the art ) o poruszanym problemie
% \item  studia literaturowe \cite{bib:artykul,bib:ksiazka,bib:konferencja,bib:url} -  opis znanych rozwiązań (także opisanych naukowo, jeżeli problem jest poruszany w publikacjach naukowych), algorytmów, 
% \end{itemize}


\section{Założenia}

\quad Wymagane możliwości oprogramowania można podzielić na trzy elementy:

\begin{itemize}
    \item Łatwość użytkowania
    \item Dostępność
    \item Możliwość dokładania własnych elementów
\end{itemize}

\quad Poprzez łatwość użytkowania można rozumieć oprogramowanie zapisane zgodnie z powszechnie stosowanymi standardami (\enquote{Zen of Python}) oraz przygotowaną dokumentację wraz z przykładami wykorzystania.  

\quad Paczka powinna być dostępna poprzez menedżer \textbf{pip}, dzięki czemu użytkownik ma możliwość instalacji z repozytorium paczki przy pomocy jednej komendy.

\quad Użytkownik powinien mieć możliwość stworzenia własnego modelu rozpoznającego gesty, na przykład przy pomocy interaktywnej instrukcji.

\section {Sformułowanie problemu}
% \quad Problem można podzielić na trzy części. Każda z nich odpowiada za wybraną część funkcjonalności biblioteki. 

% \begin{itemize}
%     \item Wyznaczenie pozyci dłoni, odległości pomiędzy wybranymi palcami oraz jej kąta obrotu. 
%     \item Rozpoznawanie gestów dłoni w dwóch trybach: prostym (parę dostępnych gestów) oraz zaawansowanym, który rozoznaje alfabet języka migowego. 
%     \item Dostępność biblioteki poprzez platfromę PyPi. 
% \end{itemize}

\quad Stworzenie modułu będzie wymagało napisania klasy pozwalającej na rozpoznanie elementów charakterystycznych dłoni oraz przetworzenia obrazu. Obraz będzie pochodził z kamerki internetowej, który zostanie odpowiednio przetworzony z wykorzystaniem funkcji dostępnych poprzez bibliotekę OpenCV. Przetworzony obraz zostanie wykorzystany przez metody biblioteki MediaPipe, która pozwoli na rozpoznanie elementów charakterystycznych dłoni. W napisanej klasie zostaną zaimplementowane metody, które pozwolą na rozpoznanie typu dłoni (prawa, lewa), odległości między paliczkiem palca wskazującego oraz kciuka, oraz obrotu dłoni. 

\quad Kolejnym elementem jest wygenerowanie modelu matematycznych klasyfikujących gesty dłoni. W tym celu zostanie wykorzystana biblioteka SciKit-Learn wraz z dostępnymi poprzez nią algorytmami ucznia maszynowego. Odpowiednio zebrane dane pozwolą na przeprowadzenie procesu ucznia dla kilku wybranych algorytmów. 

% \quad Najważniejszym elementem pracy jest powyżej wymieniony ostatni punkt listy. Dzięki wykorzystaniu platformy PyPi deweloperzy będą mogli przy pomocy programu \textbf{pip} za pomocą jednej komendy zainstalować paczkę wraz ze wszystkimi wymaganymi zależnościami. 

% \section{Podobne rozwiązania}

% \chapter{Założenia projektowe}

% \section{Budowa modułu}
% Moduł został napisany przy pomocy paradygmatu programowania
% obiektowego, co pozwala na przystępne wykorzystanie biblioteki w dowonych 
% projektach wymagających osbługi gestów. 

% \section{Dostęp}
% Całość projektu będzie dostępna na platformie GitHub wraz z możliwością 
% pobrania przy pomocy programu pip ze zdlanego repozytorium PyPi, dzięki czemu 
% pozowli to na sprawne i proste wykorzystanie modułu w dowolnym 
% projekcie. 