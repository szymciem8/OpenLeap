\chapter{Analiza tematu}

\section{Założenia}

\quad Wymagane możliwości oprogramowania można podzielić na trzy elementy:

\begin{itemize}
    \item Łatwość użytkowania -- Poprzez łatwość użytkowania można rozumieć oprogramowanie zapisane zgodnie z powszechnie stosowanymi standardami (\enquote{Zen of Python}) oraz przygotowaną dokumentację wraz z przykładami wykorzystania. 
    \item Dostępność -- Paczka powinna być dostępna poprzez menedżer \textbf{pip}, dzięki czemu użytkownik ma możliwość instalacji z repozytorium paczki przy pomocy jednej komendy.
    \item Możliwość dokładania własnych elementów -- Użytkownik powinien mieć możliwość stworzenia własnego modelu rozpoznającego gesty, na przykład przy pomocy interaktywnej instrukcji.
\end{itemize}


\section {Sformułowanie problemu}

\quad Stworzenie modułu będzie wymagało napisania klasy pozwalającej na rozpoznanie elementów charakterystycznych dłoni oraz przetworzenia obrazu. Obraz będzie pochodził z kamery internetowej, który zostanie odpowiednio przetworzony z wykorzystaniem funkcji dostępnych poprzez bibliotekę OpenCV. Przetworzony obraz zostanie wykorzystany przez metody biblioteki MediaPipe, która pozwoli na rozpoznanie elementów charakterystycznych dłoni. W napisanej klasie zostaną zaimplementowane metody, które pozwolą na rozpoznanie typu dłoni (prawa, lewa), odległości między paliczkiem palca wskazującego oraz kciuka, oraz obrotu dłoni. 

\quad Kolejnym elementem jest wygenerowanie modelu matematycznego klasyfikujących gesty wykonywane przez dłonie. W tym celu zostanie wykorzystana biblioteka SciKit-Learn wraz z dostępnymi poprzez nią algorytmami ucznia maszynowego. Odpowiednio zebrane dane pozwolą na przeprowadzenie procesu uczenia dla kilku wybranych algorytmów. 
